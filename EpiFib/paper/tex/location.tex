\section{Data Privacy} \label{sec:data-privacy}

\onehalfspacing

\subsection{Location}

Leveraging the crowdsourcing network in emergencies does not necessarily require the system to process or store user locations. For instance, if Person $X$ in Area $A$ is looking for an autoinjector and sends a request to the service without any location information, the system can simply send contact info for Person $X$ to all other users, or contact info for all other users to Person $X$, and allow the location sharing to happen via non-system communication methods. However, even with a very small number of users, this is highly impractical and provides very little benefit. 

On the other hand, if the system collects user locations, then the service can  become more streamlined and useful. Suppose the service does not persistently store any user location information. If Person $X$ in Area $A$ sends a request to find an autoinjector with their current location information, then all other mobile devices on the network can be notified of that location and alert the user if they are within a certain proximity of that location. Then, user can opt to respond if they are able. This method is much better than sharing no location information with the service at all; yet, if there are $n$ users on the network, $n-1$ users will always be notified for every emergency, regardless of their location or proximity. This protocol creates many useless notifications and still requires $n-1$ client apps to check if they are nearby the emergency.

Persistent system knowledge of all users' locations, however, can improve the service drastically: when Person $X$ requests an autoinjector, the system can send notifications only to users within a reasonable proximity to Person $X$. Users who receive a notification will immediately know they are within a reasonable distance of someone in need of an autoinjector and can opt to respond. This protocol will likely be taken seriously by users and does not require any unreasonable communication of users' locations. However, storing users' locations comes at a price: privacy. Users might be reluctant or unwilling to have their locations stored in or processed by the system. If the system did store user locations and became compromised, location information would be exposed. In the next section, I discuss different methods to store user locations while providing some degree of anonymity in the event of a security breach and still retaining the usefulness of the location data.

\subsubsection{Location Privacy}

In the context of protecting privacy for the public or private release of any data set for purposes of research, Sweeney \cite{sweeney} presents a model known as $k$-anonymity, where any data set released ``provides $k$-anonymity protection if the information for each person contained in the release cannot be distinguished from at least $k -1$ other individuals in the data set" \cite{sweeney}. This model, while considered weak in some aspects, can be adopted to the storage and processing of user information in network databases. If a database is $k$-anonymized and later breached, then the users whose information was compromised still retain some form of privacy: namely, anonymity amongst $k$ other users.

Gedik and Liu \cite{gedik} describe a $k$-anonymity model for protecting user privacy in mobile systems that allows for each user to indicate the level of $k$-anonymity desired for each message sent to a particular service. The $k$-anonymity is achieved by either decreasing location accuracy until $k-1$ other users share the same spatial location or delaying message processing until $k-1$ other messages have been received from the same spatial region.

Cheng et al.~\cite{Cheng2006} present a different scheme that uses a more probabilistic approach. They argue that $k$-anonymity may not be used if there are fewer than $k$ users in the system, and that even if there are more than $k$ users, ``they may span a large area over an extended time period, in which case the cloaked location can be large and cause a severe degradation of service quality." They instead provide a framework that stores cloaked geometric areas for each user, where the users' true location is somewhere in the area stored. The framework lets each user decide on the size and boundaries of the cloaked location stored, and the larger the area, the more anonymity. $k$-anonymity may be achieved as a side effect in this framework if cloaked locations overlap, but it does not dictate how locations are anonymized. Location-based search queries find cloaked areas that overlap with a given query range, and queries have a runtime in the worst case of $O(e^2\log e)$ and best case of $O(m+e)$, where $e$ is the number of sides of the geometric shapes stored and $m$ is the number of sides of any polygon used for the query computation.

These frameworks (and others) have their advantages, but neither is the best fit for this application. $k$-anonymity is not a good option for reasons described by Cheng et al.~\cite{Cheng2006}: if there are too few users within a given region, then the size of the cloaked location stored would be too large to provide useful data to the service. Similarly, letting each user specify the size of the cloaked region might cause the service to run into problems as discussed previously if they specify a large area. Instead, I present the following location privacy contract to the users of the application and an implementation differing from those above to achieve it.

\subsubsection{Data Privacy Contract} \label{sec:privacycontract}

\begin{enumerate}
    \item A user's exact location will never be stored in any system database.
    \item For any user $u_i \in U$, let the stored location of that user in the system be $L(u_i)$. The granularity of $L(u_i)$ will be no better than 1.4 $\text{km}^2$, where $\text{Area}(L(u_i)) \geq 1.4 \ \text{km}^2 \ \ \forall \ i$, and the true location of user $u_i$ is uniformly distributed across $\text{Area}(L(u_i))$.
    \item Only the most recently received location will be stored for each user, and location history will never be stored.
    \item User locations will be stored separately from other user information, including user IDs. The IDs for each user location entry will be hashed using a secure hash function. An adversary would need the user ID, hash function, and entire location table to find a specific user's entry.
\end{enumerate}

This contract ensures that in the event of a data breach or compromise, even if an adversary had full knowledge of the source code, any user locations would only be known within 1.4 km$^2$ of the true locations. If an adversary only had access to the locations table, then the hashed user IDs would prevent them from knowing which cloaked location belonged to whom without knowledge of user IDs and the secure hash function. Additionally, this granularity of locations stored allows the service to retain the functionality it needs. Next, I present a method for implementing this contract.

\subsubsection{Mercator Projection Cloaking}

To store the location of the user while maintaining no more than 1.4 km$^2$ granularity, we opt to take a grid-based approach. We conceptually divide the Earth into a grid, where each square in the grid has area 1.4 km$^2$. When the system receives a latitude and longitude point from a user, it maps this latitude and longitude on to the grid and stores which square the user reported. As distances between adjacent meridians and parallels on the globe are not constant, the mapping is non-trivial but can easily be solved with a Mercator projection. As Weisstein \cite{mercator} shows, for any (latitude, longitude) pair $(\lambda, \phi)$, where $\lambda$ and $\phi$ are measured in radians, $(\lambda, \phi)$ can be mapped to a Cartesian $(x, y)$ point such that $x \in (-1, 1)$ and $y \in (-1, 1)$, where
\begin{align*}
    x &= \lambda - \lambda_0\\
    y &= \ln \tan \left(\frac{\pi}{4} + \frac{\phi}{2} \right)
\end{align*}
and $\lambda_0$ represents the reference meridian and is equal to 0 when used with standard latitude and longitude systems. To scale these components into units of kilometers, we multiply each component by the radius of the Earth, $R_E$, and to decrease the granularity in the $x$ and $y$ directions, we can take the floor of each component divided by $\alpha$ and $\beta$ respectively, where $\alpha$ and $\beta$ are the granularities desired in each direction. Based on our privacy contract, $\alpha$ and $\beta$ will each be 1.4 km. Now, our cloaked components take the form
\begin{align*}
    x_{\text{cloaked}} &= \left\lfloor \frac{R_E\lambda}{\alpha} \right\rfloor \\
    y_{\text{cloaked}} &= \left\lfloor \frac{R_E}{\beta} \ln \tan\left( \frac{\pi}{4} + \frac{\phi}{2} \right) \right\rfloor
\end{align*}
Note that any $(\lambda, \phi)$ pair such that $\phi = \pm\pi$, the $y$ component value is undefined. To store any user location in or query for users that are within proximity to a particular location, I present Algorithms \ref{mer-cloak}, \ref{mer-store}, and \ref{mer-query}. In Algorithm \ref{mer-query}, $U$ is the set of all users and $\tau$ is used as a constant to eliminate any entries in the database that have expired.

The runtime for \textsc{MPCloak} is $O(1)$, assuming the multiplications and division run in constant time. Under most systems and $(\lambda, \phi)$ values, this a reasonable approximation. The runtime of \textsc{StoreLocation} and \textsc{QueryLocation} depend on the storage implementation. Assuming a binary heap and that all of our data fits in memory, we achieve $O(n\log n)$ for both, where $n$ is the total number of users.

\begin{algorithm}[h!] \label{mercatoralgo}
\caption{Mercator Projection Cloaking} \label{mer-cloak}
\begin{algorithmic}[1]
\Procedure{MPCloak}{$(\lambda, \phi)$}
\If{$\phi$ is $\pm\pi$}
\State reject
\EndIf
\State $x \gets \lfloor R_E\lambda / \alpha \rfloor$
\State $y \gets \lfloor R_E \ln \tan\left( \frac{\pi}{4} + \frac{\phi}{2}\right) / \beta \rfloor $
\State return $(x, y)$
\EndProcedure
\end{algorithmic}
\end{algorithm}

\begin{algorithm}[h!] \label{mercatoralgostore}
\caption{Mercator Projection Storage} \label{mer-store}
\begin{algorithmic}[1]
\Procedure{StoreLocation}{$u_\text{id}$, $(\lambda, \phi), t$}
\State $(x,y) \gets \text{\scshape MPCloak}((\lambda, \phi))$
\State store$(h(u_\text{id}), (x, y), t)$
\EndProcedure
\end{algorithmic}
\end{algorithm}

\begin{algorithm}[h!] \label{mercatoralgoquery}
\caption{Mercator Projection Query} \label{mer-query}
\begin{algorithmic}[1]
\Procedure{QueryLocations}{$(\lambda, \phi)$}
\State $(x,y) \gets \text{\scshape MPCloak}((\lambda, \phi))$
\State $R \gets u_i \in U$ : $h(u_i).\text{location} = (x,y)$
\State $S \gets u_i \in U$ : $h(u_i).t - t_{\text{cur}} < \tau$
\State retrieve$(R \cap S)$
\EndProcedure
\end{algorithmic}
\end{algorithm}

Using $\alpha$ and $\beta$ values of 1.4 km with these algorithms fulfills point 2 of the location privacy contract. Points 1, 3, and 4 depend strictly on implementation, and care to ensure they are met is needed.

One way to ensure that the granularity requirement is met by an application is to move the location cloaking logic to a trusted third party or perform it on the mobile client. The third party can handle cloaking and storage, and the application can simply make queries and handle requests to and from the third party. This may or may not always be practical for end user experience depending on the implementation, and in this implementation, the cloaking and storage is performed server-side while respecting abstraction barriers.

