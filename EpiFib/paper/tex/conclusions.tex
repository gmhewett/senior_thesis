\section{Conclusions and Future Work} \label{conclusions}

\onehalfspacing

This system allows users to quickly and efficiently find autoinjectors and defibrillators near them by leveraging crowdsourcing and strategically-placed smart containers. To create the system, the service utilizes a user-facing mobile app, Internet-connected smart containers, and a cloud service for processing and handling data and communications. Based on population and prescription data, crowdsourcing autoinjectors is feasible given reasonably high population densities, and the presence of smart containers makes the network useful even when no registered users are nearby. Under the hood, the service processes user locations while preserving a reasonable level of privacy and an added level of comfort to the user. 

Initial prototypes of the mobile app and smart containers are useful and promising. To prepare forreal deployment, a user-friendly and intuitive design is needed for the app. Hardware components would need to be implemented on a printed circuit board (PCB) and optimized for mass production. In addition, creating a ``retro-fit'' system for integrating the smart container circuitry into existing AED cabinets would allow for optimal placement and integration into communities.

Working with 911 call centers also has promising indications. By integrating with existing systems, this service can more easily integrate with current user behavior during emergencies while moving some liability issues to the call centers.

Other future steps include conducting user interviews. Putting this app in the hands of users under controlled circumstances and receiving feedback would be invaluable to the development. Additionally, partnerships with local hospitals and allergy clinics would provide insight into the perspectives of experts on cardiac arrest and anaphylactic shock and a line of communication with potential customers.
