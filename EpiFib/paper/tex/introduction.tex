\section{Introduction}

\onehalfspacing

\subsection{Motivation}

The field of medical devices has continued to shape the practice and perception of medical care since the invention of the stethoscope and thermometer and into the age of the CAT scan and prosthetic limbs. The world's premier hospitals are home to highly sophisticated technology capable of efficiently saving lives that could not have been saved in recent history \cite{advhosp}. However, despite these advancements, many life-saving medical devices often remain inaccessible in times of need. For instance, autoinjectors used to administer epinephrine during anaphylactic shock are needed immediately to counteract an allergic reaction for those who suffer from food allergies \cite{epifast}; yet, if someone in need does not have an autoinjector on their person, they waste crucial time waiting on an ambulance or driving to a hospital. Emergency defibrillators recognize heart arrhythmia and can save a person's life during cardiac arrest before an ambulance has time to arrive on the scene \cite{Kerber1677}. Many buildings contain public containers with access to defibrillators in the case of an emergency, but this requires someone nearby the emergency finding the container without knowing whether one is even nearby. In the cases of anaphylactic shock and cardiac arrest, the medical technology is available for saving a life, but the device may simply be physically inaccessible during the time of need.

Compared with the inaccessibility of medical devices such as these, the Internet and crowdsourcing have made many other goods and services dramatically more accessible. Ride-sharing mobile apps such as Uber and Lyft have dramatically increased the accessibility of cars in recent years. The apps allow a user to find or request a ride nearby within seconds instead of looking for a taxi or walking to a bus stop. Websites such as Kickstarter and Go Fund Me have removed the door-to-door hassle of fundraising for a startup or personal venture and brought investors and donors directly to users---faster than any door-to-door campaign ever could. A simple PA system allows an employee at an airline terminal to quickly find a volunteer to check a bag if the overhead bin space is sure to be full by asking everyone all at once instead of individually. Innovations such as these can provide quick and easy access to nearby services and information without the need to physically search any physical area.

\subsection{Contributions}

The contributions of this thesis are...

The goal of this paper is to explore how the Internet and crowdsourcing can best be used to improve the accessibility of certain life-saving medical devices in time-sensitive emergency situations. We will show that, by using both web and mobile apps and containers connected to the Internet, it is possible to create such a network that significantly improves the accessibility of these devices. In Section \ref{background}, we provide background on why connecting these medical devices is important, what others have already done to address this inaccessibility problem, and which medical devices we will use for the purposes of our design and why. Section \ref{numbers} gives detailed statistical analysis revealing why a crowdsourcing network is possible for finding and obtaining a medical device quickly via a mobile app. In this section, we also discuss why using strategically-placed, Internet-connected containers are favorable for creating a more useful network of devices that cannot be achieved through crowdsourcing alone.

The overall design goals and specifications are developed in Section \ref{specs}. We will discuss why the network needs a mobile platform to facilitate peer-to-peer communication in order to locate people nearby with an available device and, more specifically, what features the app must have to be functional and secure. Additionally, in this section we will discuss why connected containers that house devices are needed to improve the usefulness of the network. Requirements and specifications for the containers and the web app that will administer them will be given here, as well as discussion as to how the containers, web app, and mobile app should interact. Section \ref{implementation} will present our approach to meet the design specifications developed in Section \ref{specs}.

In Section \ref{evaluation}, we will give an evaluation as to how well our implementation met the design specifications and addressed the overall issue of medical device inaccessibility. In general, we found that our approach was successful and could be deployed after addressing some regulatory and legal issues. Related work will be discussed in Section \ref{related}, and Section \ref{conclusions} details our conclusions and plans for future work. 

