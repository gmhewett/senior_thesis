\section{Introduction}

\onehalfspacing

\subsection{Motivation}

Medical devices have shaped the practice and perception of medical care since the invention of the stethoscope and thermometer and into the age of the CAT scan and prosthetic limbs. The world's premier hospitals are now home to highly sophisticated technology capable of efficiently saving lives that could not have been saved in the past \cite{advhosp}. However, despite these advances, many life-saving medical devices often remain inaccessible in times of need. For instance, autoinjectors used to administer epinephrine during anaphylactic shock are needed immediately to counteract an allergic reaction for those who suffer from food and/or insect allergies \cite{epifast}; yet, if someone in need does not have an autoinjector on their person, they waste crucial time waiting for an ambulance or driving to a hospital. Emergency defibrillators recognize heart arrhythmia and can save a person's life during cardiac arrest before an ambulance has time to arrive on the scene \cite{Kerber1677}. Many buildings contain public containers with access to defibrillators in the case of an emergency, but this requires someone near the emergency to find the container without knowing whether one is even nearby. In the cases of anaphylactic shock and cardiac arrest, the medical technology is available for saving a life, but the device may simply be physically inaccessible during the time of need.

Contrasted with the inaccessibility of medical devices such as these, crowdsourcing and the Internet have made many other goods and services dramatically more accessible. Ride-sharing mobile apps such as Uber and Lyft have dramatically increased the accessibility of transportation in recent years. These apps allow a user to find or request a ride nearby within seconds instead of looking for a taxi or walking to a bus stop. Websites such as Kickstarter and GoFundMe have removed the door-to-door hassle of fundraising for a startup or personal venture and brought investors and donors directly to users---faster than any door-to-door campaign ever could. A simple PA system allows an employee at an airline terminal to quickly find a volunteer to check a bag if the overhead bin space is sure to be full by asking everyone all at once instead of individually. Innovations such as these can provide quick and easy access to nearby services and information without the need to physically search for them.

The contribution of this thesis is leveraging crowdsourcing and the Internet to design and implement a platform for quickly finding autoinjectors and defibrillators in emergencies. I show that, by using the web, mobile apps, and Internet-connected smart containers, it is possible to create a network that significantly improves the accessibility of these devices. Specifically, a mobile app is used for both crowdsourcing autoinjectors and locating nearby smart containers that house autoinjectors and defibrillators. 

\subsection{Scope}

In Section \ref{background}, I provide background on what others have already done to address this inaccessibility problem, and Section \ref{sec:population} gives a detailed statistical analysis revealing why a crowdsourcing network is possible for finding and obtaining an autoinjector quickly via a mobile app.

Data privacy considerations are important to consider for a service providing access to medical devices, and I discuss these in section \ref{sec:data-privacy}. Section \ref{sec:overview} provides a high-level overview of the system design and describes the user experience, while sections \ref{sec:hard-design} and \ref{sec:soft-design} reveal the detailed hardware and software design goals and implementations. In these sections, I discuss what the service needs to allow users to quickly find autoinjectors and defibrillators in emergencies and, specifically, what features the app and smart containers must have to be functional and secure.

In Sections \ref{sec:hard-eval} and \ref{sec:soft-eval}, I evaluate how well the hardware and software implementations met the design specifications and addressed the overall issue of autoinjector and defibrillator inaccessibility. I found that my approach was successful and could be deployed to production after addressing additional legal and manufacturing matters. Section \ref{conclusions} provides conclusions and plans for future work. 

The service described in this paper provides users with a mobile app for finding, sharing, or receiving medical devices. As such, the deployment of this service would be subject to regulatory and compliance scrutiny. Legal considerations are beyond the scope of this paper and not needed for internal testing and development, but these considerations would be necessary and prudent for actual product deployment.

