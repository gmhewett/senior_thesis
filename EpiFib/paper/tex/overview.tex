\section{Design Overview} \label{sec:overview}

\onehalfspacing

The system design consists of three main components:
\begin{enumerate}
    \item A user-facing mobile app for users to find autoinjectors and defibrillators.
    \item Strategically-placed smart containers that house autoinjectors and defibrillators.
    \item A cloud presence for handling data and communication between users and smart containers.
\end{enumerate}
These parts work together to ensure that users can find autoinjectors and defibrillators near their location quickly and efficiently. In the following sections, I give an overview of the user experience for each of these components and how they interact during usage. Sections \ref{sec:hard-design} and \ref{sec:hard-design} provide detailed, technical descriptions of these components.

\subsection{Mobile App}

The mobile app is the primary interface for user interaction with this service. When a user downloads the app on their mobile device, they are required to create an account with system using a unique email address and password. During the onboarding process, the user is required to give the app permission to deliver remote notifications and monitor the device's location. Remote notifications allow the service to inform the user if they are nearby an emergency and can possibly provide assistance. The location of the user is monitored and uploaded to the system's cloud presence using the methods described in Section \ref{sec:data-privacy}. Monitoring the locations of all users allows the service to send targeted notifications to those near emergencies.

Once a user has been onboarded and their location is being monitored, the user spends little time interacting with the app. If a user is experiencing or witnessing an emergency that requires an autoinjector or defibrillator, the user initiates the ``Find an Autoinjector'' or ``Find a Defibrillator'' functions of the app, prompting the user to always call 911-services before proceeding. If the emergency request is for an autoinjector, the cloud service notifies all users nearby the emergency in hopes that they can help. Regardless of the emergency type, the mobile app displays any smart containers housing requested devices nearby and any users who have opted to provide assistance. If a user wants to navigate to a smart container, the app provides directions. If a user has responded to help, the app shows who is on the way and where they are.

If a user receives a notification that someone nearby needs an autoinjector, the app displays the location of the emergency to the user. The user then has the option to confirm that they have an autoinjector and are willing to provide it to the user in need. The app then provides the user directions to the emergency.

\subsection{Smart Containers}

As discussed in detail in Section \ref{sec:hard-formfactor}, the smart containers are meant to resemble, or retro-fit to, existing AED defibrillator cabinets (see Figure \ref{fig:existing-aed}). These containers connect to the internet and house the autoinjectors and defibrillators in easy-to-see and easy-to-access locations. During an emergency, a user can navigate to a container via the mobile app. When a user sends their intention to find a container to the cloud, a signal is sent to the appropriate smart container indicating that someone is looking for it. The smart container then activates its LEDs and speaker to flash red lights and sound an alarm to help the user find the container.

When a container is not in alarm mode, it periodically sends telemetry data to the cloud to make sure it is working properly. Because autoinjectors are sensitive to different temperatures, the device monitors and sends temperature and humidity information to the cloud to ensure that no thresholds are breached. In addition, any time someone opens the door of the container, the container lights up and a message is sent to the cloud.

\subsection{Cloud}

Processing and handling data and communications between users and smart containers requires many cloud-based components. A web server exposes REST API endpoints that handle all communications between mobile devices and the cloud. Message queues handle data sent to the cloud from smart containers to ensure delivery even when the number of containers get large. Multiple databases store all system data. This includes device information telemetry messages, and statuses for each smart container, as well as account information, location, and emergency status information for each user. The cloud processes requests and messages from users and smart containers to manipulate the data and provide the backend of the service.

In addition to these functions that drive user experience, the cloud also provides an administrator web dashboard for managing and adding smart containers on and to the network. In addition, a simulator provides the ability to simulate smart containers for testing with the mobile app before they have been deployed.

\subsection{Alternative User Experience} \label{sec:alt-user-ex}
 
A mobile app that aims to change behavior during emergencies asks a lot of its users. The mobile app implemented stresses that calling emergency services should always be the first step in an emergency, but often behaviors aren't rational in these situations. A different user experience model involves incorporating emergency operators in the system. 

Instead of using the app to find users or smart containers nearby in an emergency, a user in need calls 911 as usual, and the emergency service operator uses the network infrastructure to direct nearby users to the emergency or send the location of nearby smart container to the calling-user's mobile device. The emergency services operator has full knowledge of all ambulances en route and can provide the best action directions to the user based on all knowledge available. This model does not change user behavior during emergencies themselves and offers a more reliable and familiar experience. Letting emergency operators dispatch nearby users was adopted by the PulsePoint app discussed in Section \ref{sec:prior-art}.

While promising, this model is difficult to build and demonstrate in the amount of time allotted for paper due to the coordination and infrastructure implementations needed to interact with a 911 provider. Instead, the platform utilizes the mobile app described above to let users find nearby users and smart containers themselves. The mobile app, smart containers, and cloud-based services can, however, be easily be adopted to incorporate the emergency operator model described here.


 